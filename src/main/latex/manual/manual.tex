% Copyright (c) 2013 by the University of Waikato, Hamilton, NZ.
% This work is made available under the terms of the Creative Commons
% Attribution-ShareAlike 3.0 license,
% http://creativecommons.org/licenses/by-sa/3.0/.
%  Version: $Revision: 4890 $

\documentclass[a4paper]{book}

\usepackage{wrapfig}
\usepackage{graphicx}
\usepackage{hyperref}
\usepackage{multirow}
\usepackage{scalefnt}
\usepackage{tikz}

% watermark -- for draft stage
\usepackage[firstpage]{draftwatermark}
\SetWatermarkLightness{0.9}
\SetWatermarkScale{5}

% Copyright (c) 2009 by the University of Waikato, Hamilton, NZ. 
% This work is made available under the terms of the 
% Creative Commons Attribution-ShareAlike 3.0 license, 
% http://creativecommons.org/licenses/by-sa/3.0/. 
%
% Version: $Revision: 2916 $

\newenvironment{tight_itemize}{
\begin{itemize}
  \setlength{\itemsep}{1pt}
  \setlength{\parskip}{0pt}
  \setlength{\parsep}{0pt}}{\end{itemize}
}

\newenvironment{tight_enumerate}{
\begin{enumerate}
  \setlength{\itemsep}{1pt}
  \setlength{\parskip}{0pt}
  \setlength{\parsep}{0pt}}{\end{enumerate}
}

% if you just need a simple heading
% Usage:
%   \heading{the text of the heading}
\newcommand{\heading}[1]{
  \vspace{0.3cm} \noindent \textbf{#1} \newline
}

\newcommand{\icon}[1]{\tikz[baseline=-3pt]\node[inner sep=0pt,outer sep=0pt]{\includegraphics[height=1.1em]{#1}};}


\title{
  \Huge{\textbf{Collective and Semi-supervised classification}} \\
} \author{
  Bernhard Pfahringer \\
  Kurt Driessens \\
  Peter Reutemann
}

\setcounter{secnumdepth}{3}
\setcounter{tocdepth}{3}

\begin{document}

\begin{titlepage}
\maketitle

\thispagestyle{empty}
\center
\begin{table}[b]
	\begin{tabular}{c l l}
		\parbox[c][2cm]{2cm}{\copyright 2013} &
		\parbox[c][2cm]{5cm}{\includegraphics[width=5cm]{images/coat_of_arms.pdf}} \\
	\end{tabular}
	\includegraphics[width=12cm]{images/cc.png} \\
\end{table}

\end{titlepage}

\tableofcontents
\listoffigures
%\listoftables

%%%%%%%%%%%%%%%%%%%%%%%%%%%%%%%%%%%
\chapter{API and classifiers}
\section{Classes and Interfaces}
Source code can be found at this location: \\
\url{}{https://code.google.com/p/collective-classification/}

All classifiers, interfaces and helper classes are located in this package:
\begin{verbatim}
	weka.classifiers.collective
\end{verbatim}

There exist two Interfaces for collective classifiers:
\begin{tight_itemize}
	\item \texttt{CollectiveClassifier} \\
	general interface for setting test/training set, number of folds in case 
	no test file is provided and the train is then split (new train = 1/folds of train, 
	new test = (folds - 1)/folds of train)
	\item \texttt{RestartableCollectiveClassifier} \\
	if the classifier needs restarts and iterations (inherits from 
	\texttt{CollectiveClassifier} and \texttt{Comparable})
\end{tight_itemize}

The following abstract classifiers all implement the CollectiveClassifer interface:
\begin{tight_itemize}
	\item \texttt{CollectiveRandomizableClassifier} \\
	derived from RandomizableClassifier, for simple collective classifiers
	\item \texttt{CollectiveRandomizableSingleClassifierEnhancer} \\
	derived from \texttt{RandomizableSingleClassifierEnhancer}, for meta classifiers 
	that take one classifier as input
	\item \texttt{CollectiveRandomizableMultipleClassifiersCombiner} \\
	derived from \texttt{RandomizableMultipleClassifiersCombiner}, for meta 
	classifiers that take several classifiers as input
\end{tight_itemize}

\section{Other classes}
\begin{tight_itemize}
	\item \texttt{CollectiveWrapper} \\
	a simple Wrapper for non-collective classifiers. used for comparison.
	\item \texttt{CollectiveInstances} \\
	used by some collective classifiers to produce the training and test file, 
	as well as used for initializing the labels and flipping them (in case 
	they are or type \texttt{RestartableCollectiveClassifier})
	\item \texttt{Splitter} \\
	splits the training set into two, according to the specified number of 
	folds and whether it should be inverted (cf. \texttt{StratifiedRemoveFolds} filter)
	\item \texttt{Flipper, \ldots, FlipHistory} \\
	used in classifiers that flip labels during iterations to improve performance
\end{tight_itemize}

\section{Classifiers}
\subsection{SimpleCollective}
\begin{verbatim}
extends CollectiveRandomizableSingleClassifierEnhancer
implements RestartableCollectiveClassifier
package weka.classifiers.collective.meta
\end{verbatim}

\begin{tight_itemize}
	\item trains with training + ``new'' test set
	\item the ``new'' test set is randomly labeled according to class distribution in training set
	\item three ways to get prediction model
	\begin{tight_enumerate}
		\item random walk (current model flips labels) / last model for prediction
		\item random walk (current model flips labels) / best model for prediction
		\item hill climbing (best model so far flips labels) / best model for prediction
	\end{tight_enumerate}
	\item three ways of comparing models
	\begin{tight_enumerate}
		\item RMS on training data
		\item RMS on test data (lower prediction of binary problem is taken as error)
		\item overall RMS (normalized train+test RMS)
	\end{tight_enumerate}
\end{tight_itemize}

\subsection{AdvancedCollective}
\begin{verbatim}
extends SimpleCollective
package weka.classifiers.collective.meta
\end{verbatim}

\begin{tight_itemize}
	\item splits test file into two T1 and T2
	\item T1 and T2 are randomly labeled according to class distribution in training set
	\item does two training runs
	\begin{tight_itemize}
		\item trains with training + T1 and tests against T2
		\item trains with training + T2 and tests against T1
	\end{tight_itemize}
	\item the root mean squared error of both tests is averaged for the overall performance
	\item prediction model and comparing models as with \texttt{SimpleCollective}
\end{tight_itemize}


\subsection{TwoStageCollective}
\begin{verbatim}
extends CollectiveRandomizableMultipleClassifiersCombiner
implements RestartableCollectiveClassifier
package weka.classifiers.collective.meta
\end{verbatim}

\begin{tight_itemize}
	\item takes 2 base classifiers
	\item 1. classifier is trained on training set 
	\item 1. classifier does the initial labeling of the test set and creates the ``new'' test set
	\item 2. classifier is built upon training and the ``new'' test set
	\item prediction model and comparing models as with \texttt{SimpleCollective}
\end{tight_itemize}

\subsection{CollectiveIBk}
\begin{verbatim}
extends CollectiveRandomizableClassifier
package weka.classifiers.collective.lazy
\end{verbatim}

\begin{tight_itemize}
	\item uses internally \texttt{weka.classifiers.lazy.IBk} to determine the 
	best K on the training set
	\item builds for all instances from the test set a neighborhood consisting 
	of K instances (the previously determined K value) from the pool of train 
	and test set (either a naïve search over the complete set of instances or 
	a \texttt{KDTree} is used to determine the neighbors)
	\item all neighbors in such a neighborhood (\texttt{CollectiveIBkNeighbors}) 
	are sorted according to their distance to the test instance they belong to
	\item the neighboorhoods are sorted according to their ``rank'', where 
	``rank'' means the difference occurrences of the two classes in the 
	neighborhood. e.g. class A appears 5 times in the neighborhood, class B 3 
	times and 2 unlabeled instances, which leads to a rank of (5 - 3) = 2.
	\item for all (still unlabeled) test instances with the highest rank the 
	class label is determined by majority vote or in case of a tie the first 
	class. due to this labeling of an instance of the test set other neighborhoods 
	change their rank (since they might contain this test instance)
	\item this is done until no further unlabeled test instances remain
	\item classification is done by returning the class label of the instance 
	that is about to be classified
\end{tight_itemize}
\textbf{Note}
\begin{tight_itemize}
	\item the \texttt{ReplaceMissingValues} filter is applied to data sets, 
	since the Euclidean-Distance returns the highest possible distance between 
	two instances if they both have a missing value at the same position. an 
	instance with at least one missing value then never has the distance 0 to itself!
	\item formerly distinct instances may become equal due to the 
	\texttt{ReplaceMissingValues} filter. In order to get the correct 
	classification for an instance the original instance is saved along the one 
	with the filter applied to. This ensures that the correct instance is 
	found during the search, and therefore the correct classification.
\end{tight_itemize}

\subsection{CollectiveWrapper}
\begin{verbatim}
extends CollectiveRandomizableSingleClassifierEnhancer
package weka.classifiers.collective.meta
\end{verbatim}

\begin{tight_itemize}
	\item takes any WEKA classifier as input
	\item used to enable comparison between ``real'' collective classifiers and 
	other WEKA classifiers
\end{tight_itemize}

\subsection{CollectiveForest}
\begin{verbatim}
extends CollectiveRandomizableClassifier
package weka.classifiers.collective.trees
\end{verbatim}

\begin{tight_itemize}
	\item uses \texttt{weka.classifiers.trees.RandomTree} as base classifier, 
	just like \texttt{weka.classifiers.trees.RandomForest}
	\item this classifier divides the test set into folds containing the same 
	number of elements
	\item if the number of folds is 0 then one has a 
	\texttt{weka.classifiers.trees.RandomForest} classifier
	\item if bagging is used (simulates the \item{weka.classifiers.meta.Bagging} 
	Meta-Classifier) the out of bag error is computed in addition
	\item the first iteration trains on the original training set and 
	generates the distribution for all instances in the test set
	\item the best instances are then added to the original training set 
	(the number of instances being chosen is the same as in a fold)
	\item the next iteration trains with the new ``enriched'' training set 
	and generates then the distributions for the remaining instances in the test set
\end{tight_itemize}

\subsection{CollectiveTree}
\begin{verbatim}
extends CollectiveRandomizableClassifier 
package weka.classifiers.collective.trees
\end{verbatim}

\begin{tight_itemize}
	\item similar to the \texttt{weka.classifiers.trees.RandomTree} classifier
	\item splits the attribute at that position that divides the current subset 
	of instances (of training and test instances) into (roughly) 2 halves
	\begin{tight_enumerate}
		\item nominal attributes \\
		\\
		{\scriptsize
		\begin{tabular}{l | r | r | r | r | r | l }
			  &  a &  b &  c &  d &  e & \\
			\hline
			p & 10 &  8 &  2 &  0 &  2 & p=pos, n=neg, ?=inst. from test set \\
			n &  1 &  2 &  7 &  0 &  0 & \\
			? & 10 & 20 &  5 & 10 &  5 & \\
			\hline
			  & 11/13 & 9/12 & 3/11 & 1/2 & 3/4 & dist: $(p+1)/(p+n+2)$ \\
			  &  1 &  2 &  4 &  3 &  2 & ranking \\
			  & 21 & 30 & 14 & 10 &  7 & overall instances \\
			\hline
		\end{tabular}
		} \\
		\\
		order according to rank and equally ranked ones are merged
		{\scriptsize
		\begin{verbatim}
->         21  37   10  14
-> split:  21  37 | 10  14
-> bins:   (a b e)  (d e)
		\end{verbatim}
		}

		\item numeric attributes \\
		find numerical value of attribute that splits the instances into 
		two equal sized halves

		\item splitting on an attribute that contains missing values: \\
		split training instances according to their class into these two 
		branches, test instances randomly according to the class-proportion 
		previously determined
	\end{tight_enumerate}

	\item the tree is stopped from growing, if one of the following 
	conditions is met
	\begin{tight_enumerate}
		\item only training instances would be covered (the labels for these 
		instances are already known!)
		\item only test instances in the leaf $\rightarrow$ taking the 
		distribution from the parent node
		\item only training instances of one class $\rightarrow$ all test 
		instances are considered to have this class
	\end{tight_enumerate}
	
	\item calculation of distributions (for complete set or a subset)
	\begin{tight_enumerate}
		\item class distribution \\
		according to the weights in the training set the weights are summed up 
		and normalized
		\item attribute distribution \\
		\textit{nominal:} for each distinct value the weights are summed up, then normalized \\
		\textit{numeric:} binary split based on median is calculated and then the weights are 
		summed up for the two bins, and finally normalized
	\end{tight_enumerate}
\end{tight_itemize}

\subsection{CollectiveEM}
\begin{verbatim}
extends CollectiveRandomizableSingleClassifierEnhancer
package weka.classifiers.collective.meta
\end{verbatim}

\begin{tight_itemize}
	\item the first step is to train a classifier on the training set
	\item the trained classifier is used to determine the distributions and 
	hence the class labels for the test set
	\item the classified test set is then duplicated and then the one set gets 
	the labels and weights of the first class, the other one of the second class; 
	together with the training set those two sets present the new training set
	\item the classifier determines the distributions always on the original test set
	\item we keep track of all the distributions we get for each instance in the 
	test set (``mean'')
	\item the new weight for an instance is calculated as follows \\
	$mean_{n+1} = q \times mean_n + (1- q) \times x_{n}$
	with $x_{n}$ as the current distribution from the classifier
\end{tight_itemize}

\noindent optional
\begin{tight_itemize}
	\item the training set is updated like the test set (i.e. splitting in two 
	and updating labels/weights)
	\item randomizing the data before it is presented to the clasifier
\end{tight_itemize}


\subsection{CollectiveWoods}
\begin{verbatim}
extends CollectiveForest
package weka.classifiers.collective.trees
\end{verbatim}

\begin{tight_itemize}
	\item works like \texttt{CollectiveForest}
	\item uses \texttt{CollectiveTree} instead of \texttt{RandomTree}
\end{tight_itemize}


\subsection{LLGC}
\begin{verbatim}
extends CollectiveRandomizableClassifier
package weka.classifiers.collective.functions
\end{verbatim}

\begin{tight_itemize}
	\item works similar to spreading activation networks
	\item based on \cite{zhou}
	\item in addition to the paper, the number of atttributes can be used as 
	an additional normalization factor in the calculation of the affinity 
	matrix, i.e., dividing the distance between two instances by 
	$2\sigma^2 \times \#attributes$ instead of $2\sigma^2$.
\end{tight_itemize}


\subsection{YATSI}
\begin{verbatim}
extends CollectiveRandomizableSingleClassifierEnhancer
package weka.classifiers.collective.meta
\end{verbatim}

\begin{tight_itemize}
	\item YATSI -- Yet Another Two Stage Idea \cite{yatsi}
	\item The first step is to train a classifier on the training set.
	\item Predictions for unlabelled instances are performed using k-nearest 
	neighbour approach, using both the training instances and the test-instances 
	as labeled by the classfier obtained in Step 1. (the instance under 
	investigation is left out)
	\item To reduce the influence of the test-set, a 
	$weight = p \times (\#train instances / \#test instances)$ is given to each 
	of the test instances. (with p a user defined parameter that can be used to 
	raise or lower the importance of the test-set)
	\item The k is a user supplied parameter, but one can also use IBk's 
	(\texttt{weka.classifiers.lazy.IBk}) internal cross-validation to determine 
	the optimal k on the training set
	\item To investigate the impact of weighting of the unlabeled instances, 
	one can also disable any weighting
	\item The \texttt{ReplaceMissingValues} filter is applied to data sets, 
	since the Euclidean-Distance returns the highest possible distance between 
	two instances if they both have a missing value at the same position. an 
	instance with at least one missing value then never has the distance 0 to itself!
\end{tight_itemize}

\subsection{FilteredCollectiveClassifier}
\begin{verbatim}
extends CollectiveRandomizableSingleClassifierEnhancer
package weka.classifiers.collective.meta
\end{verbatim}

\begin{tight_itemize}
	\item takes, just like the \texttt{weka.classifiers.meta.FilteredClassifier}, 
	a filter and a classifier as input
	\item the filter is only ``trained'' on the training set, but applied to 
	all instances, train, test and other instance objects that get passed to the 
	\texttt{FilteredCollectiveClassifier}, before they are passed on to the 
	collective base classifier
\end{tight_itemize}

\subsection{Chopper}
\begin{verbatim}
extends CollectiveRandomizableMultipleClassifiersCombiner
package weka.classifiers.collective.meta
\end{verbatim}

\begin{tight_itemize}
	\item only works on two-class-problems
	\item similar to \texttt{CollectiveForest}
	\item \textbf{first stage:} trains a base classifier on the training set 
	and determines the distributions for all instances in the test set; the 
	difference between the confidences is used as ranking criterium
	\item \textbf{second stage:} the ranked test set is divided in a number of 
	equal sized folds and the best fold is added to the current training set, 
	which is again input for another classifier
	\item Adding the best fold is done until no more instances are left or the 
	cut-off fold number is reached (i.e., only $x$ number of best folds will 
	be added, in order to keep the classifier from deteriorating with 
	not-so-good instances)
	\item if there is more than one classifier defined for the second stage, 
	then these classifiers will be used in a rotating manner
\end{tight_itemize}


\subsection{Weighting}
\begin{verbatim}
extends CollectiveRandomizableMultipleClassifiersCombiner
package weka.classifiers.collective.meta
\end{verbatim}

\begin{tight_itemize}
	\item only works on two-class-problems
	\item all base classifiers must implement the 
	\texttt{weka.core.WeightedInstancesHandler} interface
	\item \textbf{initialization step:} the first classifier is trained on the 
	training set plus the test set (all instances of this set have a weight of 0) 
	and used to set the initial labels of the test set
	\item \textbf{training step:} if there are more than 2 classifiers specified, 
	classifiers 2 to n are then used in turns to train on the combined dataset 
	(training set plus weighted and labeled test set)
	\item the training step is performed $x$ times (defined by the user), but 
	the training can also be stopped after a certain cut-off step (user-defined 
	again). In each training step the weights of the test instances are 
	calculated like this: $\#step/\#steps$. The labels of the test instances 
	are determined anew in each training step, by using the previously built 
	classifier (if step 1 this is the classifier from the initialization step).
\end{tight_itemize}

%%%%%%%%%%%%%%%%%%%%%%%%%%%%%%%%%%%
\chapter{Command-line}
The command-line for collective classifiers is similar to the ones of the 
usual WEKA classifiers. Here is a list of the general options:

{\scriptsize
\begin{verbatim}
-t <file>
	Training set

-T <file>
	Test set

-c <index>
	Class index (1-based or 'first' or 'last')
	default: last

-l <file>
	Serialized model to use, requires '-T' but does not allow '-t'

-d <file>
	Serializes a built model, not available when using cross-validation

-x <folds>
	The number of folds for cross-validation (if '-T' is omitted)
	Use -1 for leave-one-out cross-validation
	default: 10

-s <number>
	The seed value for randomization
	default: 1

-split-percentage <0-100>
	Splits the training set into train and test set with the
	specified amount set aside for training

-preserve-order
	Turns off randomization when using '-split-percentage'
\end{verbatim}
}

%%%%%%%%%%%%%%%%%%%%%%%%%%%%%%%%%%%
\chapter{Explorer}
TODO

%%%%%%%%%%%%%%%%%%%%%%%%%%%%%%%%%%%
\chapter{Experimenter}
In order to run experiments with the collective classifiers, you need to use 
the advanced interface and select the \texttt{CollectiveSplitEvaluator} rather
than the usual \texttt{SplitEvaluator}. Figure \ref{splitevaluator} shows a
cross-validation setup.

\begin{figure}[htb]
  \centering
  \includegraphics[width=10.0cm]{images/splitevaluator.png}
  \caption{Split evaluator for the collective classifiers.}
  \label{splitevaluator}
\end{figure}


%%%%%%%%%%%%%%%%%%%%%%%%%%%%%%%%%%%
% Copyright (c) 2009-2012 by the University of Waikato, Hamilton, NZ. 
% This work is made available under the terms of the 
% Creative Commons Attribution-ShareAlike 3.0 license, 
% http://creativecommons.org/licenses/by-sa/3.0/. 
%
% Version: $Revision: 4582 $

\begin{thebibliography}{999}
	% to make the bibliography appear in the TOC
	\addcontentsline{toc}{chapter}{Bibliography}

\end{thebibliography}


\end{document}
